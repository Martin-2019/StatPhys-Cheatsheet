\section{Quantum fluids}

% ######################################
\subsection*{Fermion vs. bosons}
Particles with half-integer (integer) spin are called fermions (bosons). Their
total wave function (space and spin) must be antisymmetric (symmetric)
under the exchange of any pair of identical particles.

% ######################################
\subsubsection*{Canonical ensemble}
two particles that are distributed over two states with energies 0 and $\epsilon$

\begin{equation*}
    \begin{aligned}
        Z_F &= e^{-\beta \epsilon} \; \text{ Fermi-Dirac} \\
        Z_B &= 1+ e^{-\beta \epsilon} + e^{-2\beta \epsilon} \; \text{ Bose-Einstein} \\
        Z_M &= \frac{1+ 2e^{-\beta \epsilon} + e^{-2\beta \epsilon}}{2} \; \text{ Maxwell-Boltzmann}
    \end{aligned}
\end{equation*}

% ######################################
\subsection*{Grand canonical ensemble}
Fermions:
\begin{equation*}
    z_F = 1+e^{-\beta(\epsilon-\mu)}
\end{equation*}
average occupation number $n_F$:
\begin{equation*}
    n_F = \frac{1}{e^{\beta(\epsilon - \mu)}+1} \; \text{ Fermi function}
\end{equation*}
For $T \rightarrow 0$, the fermi function approaches a step function:
\begin{equation*}
    n_F = \Theta (\mu - \epsilon)
\end{equation*}
Bosons:
\begin{equation*}
    z_B = \frac{1}{1-e^{-\beta(\epsilon-\mu)}}
\end{equation*}
average occupation number $n_B$:
\begin{equation*}
    n_B = \frac{1}{e^{\beta(\epsilon - \mu)}-1}
\end{equation*}

\begin{itemize}
    \item Fermions tend to fill up energy states one after the other
    \item Bosons tend to condense all into the same low energy state
\end{itemize}

% ######################################
\subsection*{The ideal Fermi fluid}
density of states:
\begin{equation*}
    D(\epsilon) = \frac{V}{2\pi N} \left(\frac{2m}{\hbar^2}\right)^{\frac{3}{2}} \sqrt{\epsilon}
\end{equation*}

% ######################################
\subsubsection*{Fermi energy}
\begin{equation*}
    N = \sum_{\vec{k},m_s} n_{\vec{k},m_s} = N \int_0^\infty d\epsilon D(\epsilon) n_F(\epsilon)
\end{equation*}
Limit $T \rightarrow 0$. $\mu(T=0)$ is called Fermi energy:
\begin{equation*}
    \begin{aligned}
        \epsilon_F &= (3\pi^2)^{\frac{2}{3}} \frac{\hbar^2 \rho^{\frac{2}{3}}}{2m}
    \end{aligned}
\end{equation*}

% ######################################
\subsubsection*{specific heat}
\begin{equation*}
    \begin{aligned}
        \mu &= \epsilon_F \left[1-\frac{\pi^2}{12} \left(\frac{k_B T}{\epsilon_F}\right)^2\right] \text{ for } T \ll \frac{\epsilon_F}{k_B} \\
        c_V &= \left. \frac{\partial E}{\partial T} \right|_V = N \frac{\pi^2}{3} k_B^2 D(\epsilon_F)T \\
        c_V &= N \frac{\pi^2}{2} \frac{k_B T}{\epsilon_F} k_B
    \end{aligned}
\end{equation*}

% ######################################
\subsubsection*{Fermi pressure}

\begin{equation*}
    p \overset{T\rightarrow 0}{\rightarrow}\frac{2}{5} \frac{N}{V}\epsilon_F = \frac{(2\pi^2)^{\frac{2}{3}}}{5} \frac{\hbar^2}{mv^{\frac{5}{3}}}
\end{equation*}

% ######################################
\subsection*{The ideal Bose fluid}
$\epsilon = \frac{\hbar^2 k^2}{2m}$ and conserved particle number N.

\begin{equation*}
    \begin{aligned}
        N &= \frac{N}{\lambda^3} g_{\frac{3}{2}} (z) \\
        z &= e^{\beta \mu}, \;\; \lambda = \frac{h}{(2 \pi m k_B T)^{\frac{1}{2}}} \\
        T_c &= \frac{2 \pi}{\left(\zeta\left(\frac{3}{2}\right)\right)^{\frac{3}{2}}} \frac{\hbar^2 \rho^{\frac{2}{3}}}{k_B m} \\
        E &= \frac{3}{2} k_B T \frac{V}{\lambda^3} g_{\frac{5}{2}}(z) = \frac{3}{2} k_B T N_e \frac{g_{\frac{5}{2}}(z)}{g_{\frac{3}{2}}(z)} \\
        c_V &= \frac{15}{4} k_B N \left(\frac{T}{T_c}\right)^{\frac{3}{2}} \frac{\zeta\left(\frac{5}{2}\right)}{\zeta\left(\frac{3}{2}\right)} \;( \text{ for } T \leq T_c) \\
        c_V &= \frac{15}{4} k_B N \frac{g_{\frac{5}{2}}(z)}{g_{\frac{3}{2}}(z)} - \frac{9}{4}k_B N \frac{g_{\frac{3}{2}}(z)}{g_{\frac{1}{2}}(z)}  \;( T > T_c) \\
    \end{aligned}
\end{equation*}

% ######################################
\subsection*{Classical limit}
$\mu \rightarrow - \infty$ the two grandcanonical distr. become the Maxwell-Boltzmann distr.
\begin{equation*}
    \begin{aligned}
        n_{F/B} &= \frac{1}{e^{\beta(\epsilon-\mu)}\pm1} \rightarrow e^{\beta \mu} e^{-\beta \epsilon} \\
        N &= g \frac{V}{\lambda^3} e^{\beta \mu} \\
        E &= \frac{3}{2} k_B T N
    \end{aligned}
\end{equation*}