\noindent\rule[1ex]{\textwidth/5}{1pt}
\section{The canonical ensemble}
$T=$ const , $V=$ const , $N=$ const.


% ######################################
\subsection*{Boltzmann distribution}

\begin{equation*}
    \begin{aligned}
        p_i &= \frac{1}{Z} e^{- \beta E_i} \quad \text{Boltzmann distribution} \\
        Z &= \sum_i e^{- \beta E_i} \quad \text{partition sum}
    \end{aligned}
\end{equation*}

\subsubsection*{For classical Hamiltonian systems:}
\begin{equation*}
    \begin{aligned}
        p(\vec{q}, \vec{p}) &= \frac{1}{Z N! h^{3N}} e^{- \beta \mathcal{H}(\vec{q},\vec{p})} \\
        Z_N (T,V) &= \frac{1}{N! h^{3N}} \iint d\vec{q} d\vec{p} e^{- \beta \mathcal{H}(\vec{q},\vec{p})}
    \end{aligned}
\end{equation*}
For common Hamiltonian:
\begin{equation*}
    Z_N (T,V) = \frac{1}{\lambda^{3N}N!} \int_V d\vec{q} e^{-\beta \hat{V}(\vec{q})}
\end{equation*}

% ######################################
\subsection*{Free energy}
\begin{equation*}
    F(T,V,N) = -k_B T \ln Z_N (T,V)
\end{equation*}
\begin{equation*}
    \langle E \rangle = U = - \partial_\beta \ln Z_N 
\end{equation*}

total differential:
\begin{equation*}
    dF = dE +d(TS) = -SdT - pdV + \mu N
\end{equation*}

% ######################################
\subsection*{equations of state}
\begin{equation*}
    S = - \frac{\partial F}{\partial T}, \quad p = - \frac{\partial F}{\partial V}, \quad \mu = \frac{\partial F}{\partial N}
\end{equation*}

% ######################################
\subsection*{Non-interacting systems}
$\epsilon_{ij}$ is the $j^{th}$ state of the $i^{th}$ element

\begin{equation*}
    \begin{aligned}
        Z &= \sum_{j_1} \sum_{j_2} \dots \sum_{j_N} e^{-\beta \sum_{i=1}^N \epsilon_{ij_i}} \\
        &= \left(\sum_{j_1} e^{-\beta \epsilon_{1j_1}}\right) \dots \left(\sum_{j_N} e^{-\beta \epsilon_{Nj_1N}}\right) \\
        &= z_1 \cdot \dots \cdot z_N = \prod_{i=1}^N z_i \\
        \rightarrow F &= -k_B T \sum_{i=1}^N \ln(z_i) = -k_B T \ln(Z)
    \end{aligned}
\end{equation*}

\begin{equation*}
    Z = z^N , \qquad F = -k_B T N ln(z)
\end{equation*}

% ######################################
\subsection*{Equipartition theorem}
f are the degrees of freedom.

harmonic Hamiltonian with $f=2$
\begin{equation*}
    \begin{aligned}
        \mathcal{H} &= Aq^2 + Bp^2 \\
        z &\propto \int dq dp e^{-\beta \mathcal{H}} \\
        &= \left(\frac{\pi}{A\beta}\right)^{\frac{1}{2}} \cdot \left(\frac{\pi}{B \beta}\right)^{\frac{1}{2}} \\
        &\propto \left(T^{\frac{1}{2}}\right)^f
    \end{aligned}
\end{equation*}
For sufficiently high temperture (classical limit), each quadratic term 
in the Hamiltonian contributes a factor $T^{\frac{1}{2}}$ to the partition sum \textbf{('equipartition theorem')}

\begin{equation*}
    \begin{aligned}
        F &= -k_B T \ln(z) = - \frac{f}{2} k_B T ln(T) \\
        S &= - \frac{\partial F}{\partial T} = \frac{f}{2} k_B (\ln(T)+1) \\
        U &= - \partial_\beta \ln (z) = \frac{f}{2} k_B T \\
        c_v &= \frac{dU}{dT} = \frac{f}{2} k_B
    \end{aligned}
\end{equation*}

% ######################################
\noindent\rule[1ex]{\textwidth/5}{0pt} % placeholder (RM)
\subsection*{Molecular gases}
$N$ molecules; $x$ different mode types: \\
$Z = Z_{trans} \cdot Z_{vib} \cdot Z_{rot} \cdot Z_{elec} \cdot Z_{nuc}$ \\
$Z_x = z_x^N$

\subsubsection*{Vibrational modes}
often described by the Morse potential:\\
$V(r) = E_0 \left(1- e^{-\alpha (r-r_0)}\right)^2$

An exact solution of the Schrödinger equation gives:\\
$E_n = \hbar \omega_0 \left(n + \frac{1}{2}\right) - \frac{\hbar^2 \omega_0^2}{e E_0} \left(n + \frac{1}{2}\right)^2$ \\
$\omega_0 = \frac{\alpha}{2 \pi} \sqrt{\frac{2 E_0}{\mu}} , \quad \mu = \frac{m}{2}$

For $\hbar \omega_0 \ll E_0 $ we can use the harmonic approximation:\\
$z_{vib} = \frac{e^{-\beta \hbar \omega /2}}{1- e^{- \beta \hbar \omega_0}}$\\
$T_{vib} \approx \frac{\hbar \omega_0}{k_B} \approx 6.140 K \; \text{ for } H_2$

\subsubsection*{Rotational modes}

standart approximation is the one of a rigid rotator. The moment of inertia is given as: \\
$I = \mu r_0^2 \quad T_{rot} = \frac{\hbar^2}{I k_B}$\\
$\rightarrow E_l = \frac{\hbar^2}{2I} l(l+1)$

\subsubsection*{Nuclear contributions: ortho- and parahydrogen}
$S=1 \; , z_{ortho} = \sum_{l=1,3,5,\dots} (2l +1) e^{- \frac{l(l+1) T_{rot}}{T}}$ \\
$S=0 \; , z_{para} = \sum_{l=0,2,4,\dots} (2l +1) e^{- \frac{l(l+1) T_{rot}}{T}}$

% ######################################
\subsection*{Specific heat of a solid}
\subsubsection*{Debye model}

\begin{equation*}
    \begin{aligned}
        \rightarrow \omega(k) &= \left(\frac{4 \kappa}{m}\right)^{\frac{1}{2}} \abs{\sin \left(\frac{ka}{2}\right)} \\
        \omega &= \frac{2 \pi}{T}, \quad k = \frac{2 \pi}{\lambda} \\
    \end{aligned}
\end{equation*}

Debye frequency:

\begin{equation*}
    \begin{aligned}
        \omega_D &= c_s \left(\frac{6 \pi^2 N}{V}\right)^{\frac{1}{3}} \\
        c_s &= \left. \frac{d\omega}{dk} \right|_{k=0} = \sqrt{\frac{\kappa}{m}}a
    \end{aligned}
\end{equation*}

density of states in $\omega$-space:
\begin{equation*}
    D(\omega) = 3 \frac{\omega^2}{\omega_D^3} \quad \text{for} \; \omega \leq \omega_D
\end{equation*}

count modes in frequency-space:
\begin{equation*}
    \begin{aligned}
        \sum_{modes} (\dots) = 3 \sum_k (\dots) = 3N \int_0^{\omega_D} d\omega D(\omega) (\dots)
    \end{aligned}
\end{equation*}

partition sum:
\begin{equation*}
    z(\omega) = \frac{e^{-\beta \hbar \omega/2}}{1- e^{-\beta \hbar \omega}}
\end{equation*}

\begin{equation*}
    \begin{aligned}
        \rightarrow Z &= \prod_{modes} z(\omega) \\
        \rightarrow E &= - \partial_\beta \ln(Z) = \sum_{modes} \hbar \omega \left(\frac{1}{e^{\beta \hbar \omega}-1} +\frac{1}{2}\right)\\
            &= E_0 + 3N \int_0^{\omega_D} d\omega \frac{\hbar \omega}{e^{\beta \hbar \omega}-1} \frac{3 \omega^2}{\omega_D^3}\\
        c_v(T) &= \frac{\partial E}{\partial T} \\
            &= \frac{3 \hbar^2 N}{k_B T^2} \int_0^{\omega_D} d\omega \frac{3\omega^2}{\omega_D^3} \frac{e^{\beta \hbar \omega}\omega^2}{\left(e^{\beta \hbar \omega}-1\right)^2} \\
        u &= \beta \hbar \omega \\
        c_v(T) &= \frac{9Nk_B}{u_m^3} \int_0^{u_m} \frac{e^u u^4}{\left(e^u -1\right)^2} du
    \end{aligned}
\end{equation*}
the limit for $\hbar \omega_D \ll k_B T$:
\begin{equation*}
    c_v(T) = 3Nk_B
\end{equation*}
the limit for $k_B T \ll \hbar \omega_D$: ($T_D = \frac{\hbar \omega_D}{k_B}$)
\begin{equation*}
    c_v(T) = \frac{12 \pi^4}{5} N k_B \left(\frac{T}{T_D}\right)^3
\end{equation*}

% ######################################
\subsection*{Black body radiation}

\begin{equation*}
    \begin{aligned}
        E &= \frac{4 \sigma}{c} V T^4 , \quad \sigma = \frac{\pi^2 k_B^4}{60 \hbar^3 c^2} \\
        c_v &= \frac{16 \sigma}{c} V T^3
    \end{aligned}
\end{equation*}

\begin{equation*}
    J = \frac{P}{A} = \sigma T^4 \quad \text{Stefan- Boltzmann law}
\end{equation*}
Plank's law for black body radiation
\begin{equation*}
    u(\omega) := \frac{\hbar}{\pi^2 c^3} \frac{\omega^3}{e^{\hbar\omega/(k_BT)}-1}
\end{equation*}
The Plank distribution has a maximum at:
\begin{equation*}
    \hbar \omega_{max} = 2.82 k_B T \quad \text{Wien's displacement law}
\end{equation*}